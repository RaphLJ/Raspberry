\documentclass[12pt,a4paper]{report}
\usepackage[utf8]{inputenc}
\usepackage{amsmath}
\usepackage{amsfonts}
\usepackage{amssymb}
%\usepackage{graphicx}
%\usepackage{caption}
%\usepackage{gensymb}
\usepackage{fourier}
\usepackage[francais]{babel}
%\usepackage{multirow}
%\author{LEJEUNE Raphaël}
%\title{Titre}

\usepackage[fpms]{umons-coverpage}
\umonsAuthor{Raphaël \textsc{LEJEUNE} \\ Maximilien \textsc{POTTIEZ}}
\umonsTitle{Un robot contrôlé via un Raspberry Pi}
\umonsSubtitle{Projet d'informatique}
\umonsDocumentType{Rapport de projet}
\umonsSupervisor{Sous la direction de Monsieur le Professeur\\ Mohammed \textsc{BENJELLOUN}}
\umonsDate{2015}

\begin{document}

\umonsCoverPage

\tableofcontents

\chapter{Introduction}

Décrire le but visé, l'utilité du robot.

Faire en français et en anglais !

\chapter{Matériel}

Lister le matériel utilisé, justifier sa présence

- Raspberry (pourquoi pas un Arduino, par exemple ?)
- Moteurs (quel modèle ?)
- Contrôleur (expliquer son utilité)
- Batterie
- Clé wifi
- Webcam (caractéristiques techniques)
- Réseau Wi-Fi opérationnel
- ?

\chapter{Software}

Trouver un autre titre !!!

Ici expliquer le principe des signaux PWM, des sockets, du multi-thread (même si pas utilisé... Parler des pistes envisagées)

Captures d'écran

\chapter{Qui a fait quoi}

Environ 1 page par étudiant, détailler ce qu'il a fait

\chapter{Conclusion}

Difficultés rencontrées, limitations, améliorations possibles, ...

\appendix
\chapter{1234}

Procédure d'installation ? Soit générer un exécutable, soit faire exécuter avec python 3.4 (pas 2.7)

En 2 parties : client et serveur, + configuration (adresse IP, ...)

Partie client : possibilité de donner l'image d'une machine virtuelle, correctement configurée

\end{document}

